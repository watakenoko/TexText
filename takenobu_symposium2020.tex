\documentstyle[info-utf8]{jarticle}
%
% 著者の所属が1つの研究機関の場合の様式
%
%\author{情報太郎\Email{hokkaido@ipsj.or.jp}
%\hspace{5mm}情報次郎
%\hspace{5mm}情報三郎 \\
%(北大情報科学)\contactto{札幌市北区北14条西9丁目北海道大学大学院情報科学研究科}}
%
% 著者が複数の研究機関に所属する場合の様式
%
\author{%
\begin{tabular}{ccc}
武信雄平\Email{b1016127@fun.ac.jp} &
奥野 拓 \\
\multicolumn{2}{c}{%
(公立はこだて未来大学)\contactto{北海道函館市亀田中野町116番地2公立はこだて未来大学システム情報科学部}}
\end{tabular}}

\title{観光スポットと移動経路に対する嗜好を考慮した \\
観光ルート推薦システムの構築(仮)}

\begin{document}
\maketitle

\section{はじめに}
情報処理学会北海道支部ではかねてより、情報処理北海道シンポジウムに
投稿される皆様の便宜をはかるために\LaTeX 用標準スタイルファイルを作成し、
皆様に配布することに致しました。

この原稿は情報処理学会北海道支部から配布されるシンポジウム原稿作成
要領を\LaTeX によって作成したもので、シンポジウム用標準スタイルファ
イルの使用例にもなっております。これを、よく御覧になったうえで、原稿を
\underline{締切}までに作成されますようお願い致します。

\section{どのようにして入手するか}
スタイルファイルおよび申込書の\LaTeX ファイルは、情報処理学会北海道支部
WWWホームページから入手できます。

http://www.ipsj.or.jp/sibu/hokkaido/

ブラウザから直接ダウンロードできるファイルは以下の4種類で、
漢字コード別に用意されています。
\begin{enumerate}
\item info.sty (原稿用スタイルファイル)
\item sample.tex (この原稿の\LaTeX ファイル)
\item form.tex (申込書の\LaTeX ファイル)
\item form.sty (form.texのためのマクロ)
\end{enumerate}

\begin{figure}[h]
\vspace{6.5cm}
\caption{図は標準のパラメータより柔軟な配置ができるように設定してあります}
\label{f1}
\end{figure}

\section{原稿の作成要領}

\subsection{原稿の作成}
原稿の例はsample.tex(このファイル)に入っています。これを\LaTeX にかけることで
見本原稿が出来上がります。\LaTeX 以外で原稿を作成する場合は、この見本の寸法を
参考にして上下左右のマージンをとって下さい。

\LaTeX で原稿を作成する場合は、sample.texを編集して原稿を作成して下さい。
このスタイルファイルでは、Emailコマンドを指定することで著者のメールアドレスが、また、
contacttoコマンドを指定することで連絡先の住所が、自動的に第1頁のフットノートとして記
入されます。

もし、プリンタでの印刷位置が左右のどちらかにずれてしまう場合には、info.styファ
イルの先頭にある水平オフセットの量を調整してみて下さい。

\subsection{申込書の作成}
申込書はform.texというファイルと、form.styというマクロから作られます。
このファイルをそのまま\LaTeX にかけますと、見本の申込書と同時に
申込要領が作成されます。

つぎに、form.texのコメント文字列(\%\%$\ldots$\%\%)ではさまれた部分に見本として記
入された著者名や論文題名などのデータを、各自の原稿に合うように編集して下さい。これを
\LaTeX にかけると提出用の申込書の出来上がりです。

もし、手書き用の申し込み用紙を作成したい場合は、これらの著者名や論文題名などの
データを定義するコマンドの行頭にコメントマーク(\% )を入れて\LaTeX にかけてください。

\section{まとめ}
以上、情報処理学会北海道支部で作成したスタイルファイルと申込書の使い方を説明し
てきました。このファイルを皆様の研究に役立てていただけますなら、作成者冥利に
尽きるというものです。

\begin{thebibliography}{99}
\bibitem{1} 情報処理北海道シンポジウム講演申込書, 1997.
\bibitem{2} 情報処理学会北海道支部 WWWホームページ,
http://madeira.cc.hokudai.ac.jp/RD/takai/ipsjh.html, 1997.
\end{thebibliography}
\end{document}
%
%
% End of file: sample.tex
